\documentclass[12pt, a4paper, 2.5cm]{article}
\usepackage{amsmath}
\usepackage{amsthm}
\usepackage{amssymb}
\usepackage[margin=2cm]{geometry}
\usepackage{polski}

\theoremstyle{definition}
\newtheorem{theorem}{Twierdzenie}
\newtheorem{definition}{Definicja}
\renewcommand{\proofname}{Dowód}

\title{Gry topologiczne}
\author{Adrian Bętkowski, Olaf Jankowski-Pluntke, Marta Kosz}
\date{Stan na 11.11.2025}
\begin{document}
	\maketitle
	\section{Pojęcia wstępne}
	
	TO DO:
\begin{enumerate}
\item Do bibliografii: definicje w rozdziałach "definicje" i "zupełność" zostały wzięte z notatek z topologii,warto więc wybrać z której książki korzystamy i je do tego dostosować.
\item do definicji: ciąg zbieżny
\item Czy definiować topologię, zb. otw i domkn?
 \item Czy przestrzen metryczna => topologiczna dać jako twierdzenie?

\end{enumerate}

W tym rozdziale przytoczymy często powtarzające się definicje potrzebne do pełnego zrozumienia pracy.

\begin{definition}
Metryką na zbiorze $M$ nazwiemy funkcję $d: M \times M \to [0, +\infty)$ spełniającą następujące warunki dla $a,b,c \in M$:

\begin{enumerate} 
\item $d(a, b) = 0 \iff a = b$,
\item $d(a, b) = d(b, a)$,
\item (Nierówność trójkąta) $d(a, b) \leq d(a, c) + d(c, b)$. 
\end{enumerate}

\end{definition}

\begin{definition}
Przestrzeń metryczna $(M, d)$ składa się ze zbioru $M$ oraz zadanej na nim metryki $d$. Każda przestrzeń metryczna jest również przestrzenią topologiczną.
\end{definition}

\begin{definition}
Domknięcie zb. $A$ jest to najmniejszy zbiór domknięty zawierający $A$. Oznaczać je będziemy symbolem $\overline{A}$.
\end{definition}

\begin{definition}
Przez średnicę zbioru rozumiemy:
$\rm diam(F_n) := \sup\{d(y_1,y_2)\mid y_1,y_2\in F_n\}$.
\end{definition}

\begin{definition}
	Przestrzeń Baire'a - zbiór nieskończonych ciągów liczb naturalnych z topologią produktową, ozn. $B(\omega)$.
\end{definition}

	\section{Zupełność}

\begin{definition}
Niech $(M,d)$ będzie przestrzenią metryczną. Ciągiem Cauchy'ego
nazwiemy taki ciąg $(x_n)_{n \in \mathbb{N}}$ elementów $M$, dla którego niezależnie od wyboru
$\varepsilon>0$, istnieje $N$ taki, że dla $n$ i $m$ większych lub równych
$N$ zachodzi $d(x_n,x_m) < \varepsilon$.
\end{definition}

\begin{theorem}
W dowolnej przestrzeni metrycznej każdy ciąg zbieżny jest ciągiem Cauchy'ego.
\end{theorem}

\begin{proof}
Niech $(x_n)_{n \in \mathbb{N}}$, będzie ciągiem w przestrzeni metrycznej
zbiegającym do $x$. Wobec tego dla każdego $\varepsilon>0$ istnieje taki
$N$, że dla $i \ge N$ zachodzi $d(x_i,x) < \varepsilon/2$. Wobec tego dla
dowolnych wyrazów $x_n$ oraz $x_m$ takich, że $n,m \ge N$, z nierówności
trójkąta zachodzi również
\[
d(x_n,x_m) \le d(x_n,x) + d(x_m,x) < \varepsilon/2 + \varepsilon/2
= \varepsilon.
\]
\end{proof}

\begin{definition}
Przestrzeń metryczną, która spełnia warunki z poniższego twierdzenia, nazywamy zupełną.
\end{definition}

\begin{theorem}
Następujące warunki są równoważne:
\begin{enumerate}
\item Każdy ciąg Cauchy'ego w dowolnej przestrzeni metrycznej $(M, d)$ posiada granicę.
\item (Twierdzenie Cantora) każdy zstępujący ciąg niepustych zbiorów
domkniętych $F_1 \supseteq F_2 \supseteq \ldots$ o średnicach dążących do zera
($\rm diam(F_n) \to 0$) ma niepuste przecięcie.
\end{enumerate}
\\
\end{theorem}

 \begin{proof} 
 \textbf{1 $\Rightarrow$ 2} 
 \\ Niech $F_1 \supseteq F_2 \supseteq \ldots$ będzie ciągiem zbiorów domkniętych spełniającym $\rm diam(F_n) \to 0$. Zauważmy, że możemy wybrać ciąg $(x_n)_{n \in \mathbb{N}}$, taki że dla każdego n, $x_n \in F_n$. Będzie to ciąg Cauchy’ego, ponieważ dla każdego $\varepsilon>0$ istnieje $N$, dla którego $\ d(y_1, y_2)<\varepsilon $ dla $ y_1, y_2 \in F_N$. Natomiast jeżeli $n, m \ge N$ to $x_n, x_m$ należą do $F_N$, czyli zachodzi $d(x_n,x_m) < \varepsilon$. Z założenia wiemy, że dowolny ciąg Cauchy'ego ma granicę $x$. Ponieważ w dowolnym $F_n \setminus F_{n+1}$ mamy jedynie pierwsze skończenie wiele wyrazów ciągu, to $x$ musi należeć do każdego $F_n$, czyli do ich przecięcia. 
 
 \\ 
 \textbf{2 $\Rightarrow$ 1} \\ Niech $x_n$ będzie ciągiem Cauchy'ego. Weźmy zb. domknięte $F_n = \operatorname{cl} \{x_m : m \ge n\}$. Zauważmy, że skoro $x_n$ jest ciągiem Cauchy'ego, to $\rm diam(F_n) \to 0$. Wobec tego istnieje punkt $x \in \bigcap F_n$, więc mamy $ \sup \{ d(x,y) \mid y \in F_n, x \in \bigcap F_n \} \to 0$, ale jak łatwo można zauważyć $ \sup \{ d(x, y) \mid y \in F_n, x \in \bigcap F_n \} \ge d(x,x_n)$ dla wyrazów ciągu $(x_n)_{n \in \mathbb{N}}$, zatem $x_n \to x$.

\end{proof}

	\section{Gry topologiczne}
	Gra topologiczna to w ogólności nieskończona gra pozycyjna między dwoma graczami, których kolejne ruchy polegają na wybieraniu obiektów o zadanych własnościach topologicznych, np. punktów, zbiorów otwartych/domkniętych.\\
	Niech gra toczy się między graczami I i II i niech wybierają oni w każdym kroku podzbiory przestrzeni topologicznej $X$, oznaczone odpowiednio przez $I_n$ dla gracza I oraz $J_n$ dla gracza II. Wówczas rezultatem gry jest ciąg $I_0,J_0,I_1,J_1,\dots$. Gracz I wygrywa, jeżeli ciąg ten spełnia jakąś zadaną wcześniej własność, w przeciwnym wypadku wygrywa gracz II.\\
	Strategią wygrywającą dla danego gracza nazwiemy taki sposób podejmowania decyzji w swojej turze, który gwarantuje mu zwycięstwo niezależnie od decyzji drugiego gracza.
	\section{Gra Banacha-Mazura}
	Przykładem gry topologicznej jest gra Banacha-Mazura. Rozgrywa się ona na przestrzeni Baire'a $B(\omega)$ z wyróżnionym $E\subseteq B(\omega)$. W swoim ruchu każdy gracz wybiera liczbę naturalną, konstruując w tej sposób ciąg $(m_n)^\infty_{n=0}$, gdzie elementy o indeksach parzystych wybierał gracz I a elementy o indeksach nieparzystych gracz II. Gracz I wygrywa, jeżeli $(m_n)^\infty_{n=0}\in E$, w przeciwnym wypadku wygrywa gracz II. Taką grę oznaczamy poprzez $G_{BM}(E)$.\\
	Grę Banacha-Mazura można uogólnić do dowolnej przestrzeni topologicznej $X$ z wyróżnionym podzbiorem $E\subseteq X$. Wówczas grę rozpoczyna gracz I wybierając otwarty zbiór $U_0\subseteq X$, po czym każdy gracz wybiera zbiory otwarte tak, aby utworzyć ciąg $U_0\supseteq V_0\supseteq U_1\supseteq V_1,\dots$. Gracz I wygrywa tę grę jeśli $E\cap \left(\bigcap^\infty_{n=0} U_n\right)\neq \emptyset$, w przeciwnym wypadku wygrywa gracz II. Oznaczamy tę grę poprzez $G_{uBM}(X, E)$.\newpage\noindent
	Mówimy, że gra $G_{BM}(E)$ lub $G_{uBM}(X, E)$ jest zdeterminowana, jeżeli któryś z graczy ma strategię wygrywającą.\\
	Możemy zauważyć, że jeżeli zbiór $E\subseteq B(\omega)$ jest przeliczalny, to wówczas gracz II ma strategię wygrywającą.
	\begin{proof}
		~\\
		Niech $E=\{(_lk_n)\in B(\omega):l\in\mathbb{N}\}$.\\
		Wówczas wystarczy, żeby gracz II wybrał $m_{2n+1}\neq\:_nk_{2n+1}$. Każda taka decyzja zapewnia, że $(m_{n})\neq ( _nk_n)$, zatem $(m_n)\not\in E$.
	\end{proof}
	\noindent Analogicznie można znaleźć również strategię wygrywającą dla gracza I jeżeli $B(\omega)\setminus E$ jest przeliczalny.
	\section{Aksjomat determinacji}
	Aksjomat determinacji to jeden z możliwych aksjomatów teorii mnogości, mówiący: 
	\begin{center}
		\textit{Dla każdego $E\subseteq B(\omega)$ gra $G_{BM}(E)$ jest zdeterminowana}.
	\end{center}
\end{document}

